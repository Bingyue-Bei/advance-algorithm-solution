\documentclass[11pt]{article}

% Packages for math and formatting
\usepackage{amsmath, amssymb, amsthm}
\usepackage{enumitem}  % For custom lists

% Package for pseudocode
\usepackage{algorithm}
\usepackage[noend]{algpseudocode}

% Package for code
\usepackage{listings}
\lstset{basicstyle=\ttfamily, breaklines=true}

\usepackage{titlesec} 

\titleformat{\subsection}[runin]
  {\normalfont\large\bfseries}{\thesubsection}{1em}{}

\begin{document}

\title{1. Introduction to Approximation Algorithm}
\author{Bingyue Bei}
\date{\today}
\maketitle

\section*{Exericse 1.1}
\subsection*{i} \noindent
For the maximization problem, we have definition: 
\[
\textsc{alg(I)} \geq \rho \cdot \textsc{opt(I)} \quad \forall \text { I} \text{ as input.}
\]
If $\rho > 1$, we have
\[
\textsc{alg(I)} \geq \rho \cdot \textsc{opt(I)} > \textsc{opt(I)},
\]
which contradicts the definition of an optimal solution.
\subsection*{ii} \noindent
For the minimization problem, we have definition: 
\[
\textsc{alg(I)} \leq \rho \cdot \textsc{opt(I)} \quad \forall \text { I} \text{ as input.}
\]
If $\rho < 1$, we have
\[
\textsc{alg(I)} \leq \rho \cdot \textsc{opt(I)} < \textsc{opt(I)},
\]
which contradicts the definition of an optimal solution.

\section*{Exercise 1.2}
We say that the approximation ratio $\rho$ is tight when:
\[
\rho = \inf_{I} \frac{\textsc{alg(I)}}{\textsc{opt(I)}}
\]
where the infimum is taken over all possible inputs $I$.

\section*{Exercise 1.3}
\subsection*{i} \noindent
Consider the job array [80, 80, 40]. Note here that $\text{OPT(I)} = 80 + 40 = 120$.  
The work distribution is:
\begin{itemize}
    \item Machine 1: [80, 40]
    \item Machine 2: [80]
\end{itemize}
In this case, Alg returns the optimal result, 
and is indeed a 1.05-approximation.

\subsection*{ii} \noindent
We started by building an intuition for the problem we are facing. 
Since the tasks all have very small sizes, in the optimal case, they will be distributed evenly between the two machines, 
returning a makespan which is approximately 100. So the result for a 1.05-approximation algorithm should be slightly above and very close to 105.
\newline
Here is a formal proof:  
Since $t_1, t_2, \dots, t_n \leq 10$, let $M_{i}$ be the workload on machine $o_{1}$ and $M_{2}$ be the workload on machine $o_{2}$.
We have $|M_{1} - M_{2}| \leq 20$ for an optimal solution. 
\newline
We proof this claim by constructing a proof by contradiction. Without loss of generality, assume $M_{1} \geq M_{2} + 20$. 
Remove any task, represented by $t'$, from the workload of $M_{1}$, and added it into the workload of $M_{2}$. 
The new workload for $o_{1}$, $M_{1}' = M_{1} - t' \geq M_{1} - 10$ since $t' \leq 10$. 
The new workload for $o_{2}$, $M_{2}' = M_{2} + t' \leq M_{2} + 10$. Since $M_{1} \geq M_{2} + 20$, we have $M_{1} - 10 \geq M_{2} + 10$, 
which is equivalent to $M_{1}' \geq M_{2}'$. The makespan have been reduced from $M_{1}$ to $M_{1}'$
We thus created a new solution better than the previous one which claimed to be the optimal. 
\newline
Since $M_{1} + M_{2} = 200$, and $|M_{1} - M_{2}| \leq 20$. For an optimal solution, we have $M_{1}, M_{2} \leq \frac{200 + 20}{2} = 110$.
So the optimal solution has an upper-bound of $110$, $\textsc{opt(I)} \leq 110$. 
Our algorithm is an $1.05-\text{approximation}$, meaning $\textsc{alg(I)} \leq 1.05 \cdot \textsc{opt(I)} = 1.05 \cdot 110 = 115.5$. 
\newline
Professor's claim has to be false.

\section*{Exercise 1.4}
\subsection*{A} \noindent
\subsection*{B} \noindent

\section*{Exercise 1.5}
\subsection*{i} \noindent
Given the constraints of the problem we have:
\[ 
\sum_{j=1}^{n} t_{j} \geq 500, \quad t_{1}, t_{2}, ..., t_{n} \leq 25
\]
Let $M_{i}$ be the machine whose workload determines the makespan. 
Consider the scheduling of its last job $j^{*}$ with time $t_{j^{*}}$. Before that job get scheduled, 
the total processing time of all the job which has been scheduled, represented by $load(M_{i}^{*})$ is at least $475$.
\[ load(M_{i}^{*}) = 
\frac{1}{5}(\sum_{j=1}^{j < j^{*}} t_{j}) \geq \frac{1}{5}(\sum_{j=1}^{n} t_{j}- t_{j^{*}}) 
\geq \frac{(500 - 25)}{5} = \frac{475}{5}
\]

Since that machine is where the last job is assigned to, workload of that machine has an upper-bound of $95 + 25 = 120$.

\subsection*{ii} \noindent
Considering the task array of [1, 1, ..., 1, 25], 

\end{document}